\documentclass[11pt]{article}

\usepackage[utf8]{inputenc}
\usepackage[T1]{fontenc}
\usepackage[francais]{babel}

\usepackage{parskip}

\usepackage{float}

\title{Anonymisation d'emails}
\author{Antoine Lafouasse}

\begin{document}
\maketitle

\section*{Introduction}
\addcontentsline{toc}{section}{Introduction}

L'anonymisation de données est une tâche de pré-traitement, qui consiste à
effacer d'une donnée textuelle toute mention ou élément pouvant servir à
identifier une entité du monde réel ; elle sert de fait à sécuriser une donnée
au sens où il devient plus difficile --- idéalement impossible --- de remonter
aux origines des données.

Elle joue ainsi un rôle essentiel en TAL, dans la mesure où des données
anonymisées peuvent être plus facilement exploitées en tant que corpus sans
lever de problèmes d'éthique : des données brutes, sans anonymisation, posent
le problème de l'utilisation de données potentiellement sensibles et
personnelles de personnes physiques ou morales, sans autorisation de leur part.
L'anonymisation de données en TAL est ainsi portée par une motivation autant
éthique que juridique.

Cette étude se propose de présenter une solution logicielle visant à anonymiser
un corpus écrit, consistant en données issues d'emails.

\section{Présentation des données}

Les données sur lesquelles nous travaillons proviennent entièrement du corpus
Enron, qui sont un ensemble de communications entre employés d'une entreprise
sous forme d'emails en langue anglaise.

Ces données nous sont présentées entièrement sous forme de texte brut, ce qui
veut dire que les métadonnées propres aux emails et qui dépassent le cadre du
corps du message sont encodées avec une représentation textuelle spécifique,
avec chaque champ défini sur une ligne en en-tête du message, comportant le
nom du champ et sa valuer séparés par un point-virgule (\texttt{:}).

\begin{figure}[H]
\begin{verbatim}
Message-ID: <16673858.1075846656569.JavaMail.evans@thyme>
Date: Mon, 13 Dec 1999 06:00:00 -0800 (PST)
From: susan.scott@enron.com
To: ahebert@akingump.com
Subject: kitty
Mime-Version: 1.0
Content-Type: text/plain; charset=us-ascii
Content-Transfer-Encoding: 7bit
X-From: Susan Scott
X-To: "Ann M. Hebert" <ahebert@akingump.com>
X-cc:
X-bcc:
X-Folder: \Susan_Scott_Dec2000_June2001_1\Notes Folders\All documents
X-Origin: SCOTT-S
X-FileName: sscott3.nsf

Here is a kitty for your office.

I really miss mine now that they are in Kitty Jail!
\end{verbatim}
\caption{Exemple d'email issu du corpus Enron}
\label{fig:emailex}
\end{figure}

L'exemple~\ref{fig:emailex} nous montre cet encodage des métadonnées propres aux
emails, et nous donne par là-même l'occasion d'aborder les mentions et traces
d'identification susceptibles d'apparaître dans notre corpus ; nous tâcherons
d'en établir une liste, bien que non-exhaustive.

\paragraph{Noms de personnes}

La façon la plus simple et évidente d'identifier une personne est d'utiliser son
nom. Ces données sont particulièrement courantes dans les emails dans la mesure
où les noms de l'expéditeur et des destinataires sont généralement générés
automatiquement par l'annuaire ou le serveur mail des entreprises.

Les noms peuvent également survenir sous forme de mention au sens courant du
terme, lorsqu'une personne en évoque une autre dans le corps du message. Ce type
de mention est généralement plus approximatif que les noms rapatriés
automatiquement, en ce qu'ils peuvent présenter une variation typographique, des
contractions, des surnoms ou mêmes des fautes d'orthographe --- ce qui ajoute
à la difficulté d'identification des entités nommées par la mention à laquelle
nous nous intéressons.

\paragraph{Adresses email}

Ces mentions peuvent servir d'identification plus formelle que les noms, puisque
les adresses email doivent être respectées exactement --- au détail près
qu'elles sont insensibles à la casse : les fautes d'orthographe, contractions
et alias sont ainsi beaucoup plus rares sur les adresses email que les noms
de personnes. De plus, ces adresses suivent un formatage précis qui les rendent
faciles à détecter, et dans un cadre professionnel elles comportent bien souvent
le nom et le prénom du propriétaire directement dans le texte de l'adresse.
Ainsi, les adresses email sont un élément d'identification particulièrement
puissant dans le corpus Enron qui consiste de communications professionnelles.

\paragraph{Noms d'organisations}

\paragraph{Dates et heures}

\section{Description de la procédure utilisée}

\section{Analyse des limites et résultats}

\section*{Conclusion}
\addcontentsline{toc}{section}{Conclusion}

\newpage
\tableofcontents
\end{document}