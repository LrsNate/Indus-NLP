\documentclass[11pt]{article}

\usepackage[utf8]{inputenc}
\usepackage[T1]{fontenc}
\usepackage[francais]{babel}

\usepackage{parskip}

\usepackage{float}

\title{Anonymisation d'emails}
\author{Antoine Lafouasse}

\begin{document}
\maketitle

\section*{Introduction}
\addcontentsline{toc}{section}{Introduction}

L'anonymisation de données est une tâche de pré-traitement, qui consiste à
effacer d'une donnée textuelle toute mention ou élément pouvant servir à
identifier une entité du monde réel ; elle sert de fait à sécuriser une donnée
au sens où il devient plus difficile --- idéalement impossible --- de remonter
aux origines des données.

Elle joue ainsi un rôle essentiel en TAL, dans la mesure où des données
anonymisées peuvent être plus facilement exploitées en tant que corpus sans
lever de problèmes d'éthique : des données brutes, sans anonymisation, posent
le problème de l'utilisation de données potentiellement sensibles et
personnelles de personnes physiques ou morales, sans autorisation de leur part.
L'anonymisation de données en TAL est ainsi portée par une motivation autant
éthique que juridique.

Cette étude se propose de présenter une solution logicielle visant à anonymiser
un corpus écrit, consistant en données issues d'emails.

\section{Présentation des données}

Les données sur lesquelles nous travaillons proviennent entièrement du corpus
Enron, qui sont un ensemble de communications entre employés d'une entreprise
sous forme d'emails en langue anglaise.

Ces données nous sont présentées entièrement sous forme de texte brut, ce qui
veut dire que les métadonnées propres aux emails et qui dépassent le cadre du
corps du message sont encodées avec une représentation textuelle spécifique,
avec chaque champ défini sur une ligne en en-tête du message, comportant le
nom du champ et sa valuer séparés par un point-virgule (\texttt{:}).

\begin{figure}[H]
\begin{verbatim}
Message-ID: <25013650.1075846656636.JavaMail.evans@thyme>
Date: Tue, 14 Dec 1999 10:19:00 -0800 (PST)
From: susan.scott@enron.com
To: brycoop@gte.net
Subject: Re: Hi
Mime-Version: 1.0
Content-Type: text/plain; charset=us-ascii
Content-Transfer-Encoding: 7bit
X-From: Susan Scott
X-To: brycoop@gte.net
X-cc:
X-bcc:
X-Folder: \Susan_Scott_Dec2000_June2001_1\Notes Folders\All documents
X-Origin: SCOTT-S
X-FileName: sscott3.nsf

Bryan,

Are you still having a great time?  So what is Bali like?
The only images I can summon are Gauguin's paintings, but
weren't those of Tahiti?

Things are pretty good here.  Lots of Christmas parties,
lots of Christmas cheer, etc.  Not a whole lot going on at
the office, for a change. Really, though, I think I would
much rather be in Bali.

Take care.

S.
\end{verbatim}
\caption{Exemple d'email issu du corpus Enron}
\label{fig:emailex}
\end{figure}

L'exemple~\ref{fig:emailex} nous montre cet encodage des métadonnées propres aux
emails, et nous donne par là-même l'occasion d'aborder les mentions et traces
d'identification susceptibles d'apparaître dans notre corpus ; nous tâcherons
d'en établir une liste, bien que non-exhaustive.

\paragraph{Noms de personnes}

La façon la plus simple et évidente d'identifier une personne est d'utiliser son
nom. Ces données sont particulièrement courantes dans les emails dans la mesure
où les noms de l'expéditeur et des destinataires sont généralement générés
automatiquement par l'annuaire ou le serveur mail des entreprises.

Les noms peuvent également survenir sous forme de mention au sens courant du
terme, lorsqu'une personne en évoque une autre dans le corps du message. Ce type
de mention est généralement plus approximatif que les noms rapatriés
automatiquement, en ce qu'ils peuvent présenter une variation typographique, des
contractions, des surnoms ou mêmes des fautes d'orthographe --- ce qui ajoute
à la difficulté d'identification des entités nommées par la mention à laquelle
nous nous intéressons.

\paragraph{Adresses email}

Ces mentions peuvent servir d'identification plus formelle que les noms, puisque
les adresses email doivent être respectées exactement --- au détail près
qu'elles sont insensibles à la casse : les fautes d'orthographe, contractions
et alias sont ainsi beaucoup plus rares sur les adresses email que les noms
de personnes. De plus, ces adresses suivent un formatage précis qui les rendent
faciles à détecter, et dans un cadre professionnel elles comportent bien souvent
le nom et le prénom du propriétaire directement dans le texte de l'adresse.
Ainsi, les adresses email sont un élément d'identification particulièrement
puissant dans le corpus Enron qui consiste de communications professionnelles.

\paragraph{Noms d'organisations}

Les noms d'organisation sont un élément d'identification majeur au sens où ils
permettent une identification au même titre que les noms de personnes, mais ils
peuvent se révéler plus difficiles à identifier, pour des raisons diverses.

En effet, les noms d'organisation peuvent être des noms inventés de toutes
pièces, ou être pris d'un vocabulaire soit de langue étrangère, soit spécifique
à un domaine --- il devient ainsi difficile d'identifier une organisation avec
le simple critère d'un mot inconnu, puisqu'il est presque impossible de le
distinguer des autres néologismes.

Un nom d'organisation peut également être un mot très courant, qui devient
encore plus difficile à identifier lorsque le nom de l'organisation n'est
souvent que partiellement mentionné --- par exemple, \textit{Apple Inc.} n'est
souvent mentionnée que par le nom \textit{Apple}, ce qui rend la mention
difficile à identifier puisque faisant partie du vocabulaire courant. Cette
remarque s'appliquent aussi pour les organisations dont le nom forme un
acronyme, qui lui fait partie du vocabulaire courant --- souvent
intentionnellement.

\section{Description de la procédure utilisée}

\subsection{Démarche générale}

L'approche que nous avons retenue pour anonymiser ces emails est de retenir en
mémoire les mentions que nous rencontrons dans les données, de sorte à opérer
un remplacement mot-à-mot de ces mentions de manière cohérente : en gardant
une table d'équivalence entre les mentions et leur remplacement, nous sommes
certains de la stabilité de notre anonymisation. Ainsi, nous préservons
l'exploitabilité des données résultantes pour des tâches s'appuyant sur les
entités nommées, comme la résolution de coréférence.

Ainsi, notre démarche d'anonymisation s'articule en trois étapes :

\begin{itemize}
\item Segmentation de l'entrée (tokenisation)
\item Reconnaissance des entités nommées
\item Remplacement des entités nommées
\end{itemize}

\subsection{Choix de technologies}

La conception du projet a principalement été conduite par une volonté d'utiliser
au maximum des outils existants, de sorte à limiter nos développements à 
l'articulation de ces outils ensemble --- c'est particulièrement le cas pour
la reconnaissance d'entités nommées, dont l'implémentation n'est pas triviale.

Notre choix d'environnement s'est porté sur Java, qui nous garantissait non
seulement une portabilité du code, mais nous permettait aussi de mettre
rapidement en place un environnement complet avec une intégration continue et
un package managing efficace et largement utilisé --- en l'occurrence, Apache
Maven.

Notre choix pour la reconnaissance d'entités nommées s'est ainsi porté sur
le module NER de Stanford, qui fournissait non seulement des outils en interface
graphique et en ligne de commande pour l'expérimenter et en observer les
résultats, mais aussi directement la bibliothèque compilée dans une archive JAR
et un exemple d'utilisation du code ; ainsi, la documentation fournie du code
nous a séduits et conforté notre sélection. Le seul point noir que nous retenons
à cette solution est que comme l'université de Stanford ne semblait pas avoir
de serveur Nexus, nous avons du créer à la main un artefact contenant le module
de NER et l'intégrer dans notre projet Maven. Ce détail est préjudiciable en
termes de légèreté du dépôt GiT, mais n'a pas altéré la flexibilité de 
l'architecture du projet autour de Maven.

Nous avons ainsi pu nous appuyer très largement sur la puissance de ce module
--- qui est un classifieur supervisé utilisant les CRF ---
en n'ayant à ajouter qu'une simple heuristique pour reconnaître les adresses
email en plus des entités nommées --- adresses dont la fréquence est dûe à
l'aspect épistolaire de notre corpus. Pour la tokenisation nous avons opté pour
une simplification du processus en laissant la NER de Stanford opérer sa propre
tokenisation et travailler à partir de ses résultats.

\section{Analyse des résultats et limites}

\begin{figure}[H]
\begin{verbatim}
Message-ID :
Date : Tue , 14 Dec 1999 10:19:00 -0800 -LRB- PST -RRB-
From : EMAIL_0
To : EMAIL_1
Subject : Re : Hi
Mime-Version : 1.0
Content-Type : text/plain ; charset = us-ascii
Content-Transfer-Encoding : 7bit
X-From : PERSON_2 PERSON_3
X-To : EMAIL_1
X-cc :
X-bcc :
X-Folder : \ Susan_Scott_Dec2000_June2001_1 \ Notes Folders \ All documents
X-Origin : SCOTT-S
X-FileName : sscott3.nsf

PERSON_4 ,

Are you still having a great time ?  So what is LOCATION_5 like ?
The only images I can summon are PERSON_6 's paintings , but
were n't those of LOCATION_7 ?

Things are pretty good here .  Lots of Christmas parties , lots
of Christmas cheer , etc. .  Not a whole lot going on at the
office , for a change . Really , though , I think I would much
rather be in LOCATION_5 .

Take care .

S.
\end{verbatim}
\caption{Exemple d'email issu du corpus Enron}
\label{fig:emailanon}
\end{figure}

L'exemple~\ref{fig:emailanon} est l'exemple~\ref{fig:emailex} anonymisé
par notre outil. Il nous montre que les cas simples ont été correctement
traités, à savoir les adresses email, les noms de personnes et les noms
de lieux qui ont été correctement remplacés En revanche, la signature contractée
(\textit{``S.''}) et le nom dissimulé dans un chemin de dossier ont été laissés
tels quels.

Cet exemple nous donne l'occasion d'évoquer plusieurs cas de figure qui ne sont
pas correctement reconnus par notre modèle d'anonymisation :

\paragraph{Adresses email dissimulées}

Ce type d'adresses est peu courant dans une communication intra-entreprise,
mais le devient beaucoup plus lorsqu'on touche un message ou une publication qui
est accessible sur Internet : pour éviter l'acquisition d'une adresse email
par des robots dont le rôle est de crawler internet à la recherche d'adresses,
il est occasionnel de les voir sous une forme qui n'est (idéalement) pas
détectable par une machine, mais tout de même lisible par un être humain, 
par exemple : \texttt{antoine [point] lafouasse [at] student [point] 42 [point]
fr}. On voit facilement qu'avec une adresse formatée ainsi, une expression
rationnelle visant une adresse mail sera incapable de reconnaître, et donc
d'anonymiser ce type d'adresse. On peut alors imaginer un ensemble
d'heuristiques qui permettraient de reconnaître ces dissimulations, mais comme
celles-ci ne sont pas normalisées elles sont généralement laissées à
l'imagination de l'expéditeur, leur forme peut varier de manière difficilement
prédictible.

\paragraph{Contractions de noms, surnoms}

Ces deux formes peuvent être résumées avec le même problème d'une forme
alternative qui réfère à une entité déjà connue, qui plus est avec une
expression qui ne ressemble pas forcément à une entité nommée ; nous faisons
ainsi face à deux facettes d'un problème.

Premièrement, ces formes sont susceptibles de mettre en échec le module de
reconnaissance d'entités nommées, qui pour notre cas n'a pas été spécifiquement
entraîné sur un corpus épistolaire. Stanford ayant seulement fourni un jeu de
modèles entraînés par leurs soins mais pas de solution permettant d'entraîner
nos propres modèles, cela nous contraindrait soit à abandonner cet outil
entièrement, soit penser un mécanisme complémentaire à la NER de Stanford, soit
un deuxième modèle --- partiel --- de reconnaissance d'entités nommées. Bien
entendu, l'un autant que l'autre est très coûteux en moyens et en temps.

Deuxièmement, même si on arrive à reconnaître ces expressions comme des entités
nommées, nous nous retrouvons avec plusieurs formes pouvant référer à une même
entité. Si nous opérons le remplacement tel que nous l'avons implémenté, forme
par forme, on se retrouve avec une perte d'information importante puisque les
formes des mentions elles-mêmes sont perdues et ne peuvent pas être utilisées
en résolution d'anaphore --- et ce, parfois pour des contractions simples comme
des surnoms courants, des diminutifs, ou des paraphes, qui sont des cas
élémentaires en résolution de coréférence. Nous pourrions ainsi implémenter
nous-mêmes un mécanisme de résolution de coréférence, ce qui a également un coût
élevé puisqu'il s'agit d'une tâche de sémantique computationnelle à part
entière.

\paragraph{Noms dissimulés dans les tokens}

Ce cas est particulièrement visible dans l'exemple~\ref{fig:emailanon}, avec le
nom Susan Scott dissimulé dans un chemin de dossier et qui, de fait, n'est pas
relevé par le module de NER. Nous avons de toute évidence affaire à une lacune
de tokenisation, dont la résolution est cependant très risquée puisque non
seulement la proportion de bruit dans le token est très importante, mais la
segmentation elle-même de l'entitée nommée à l'intérieur peutêtre très
variable.

Une solution pourrait être d'écrire un module complémentaire spécifique
aux chemins de dossiers, sachant que leur segmentation est libre --- obtenir un
résultat satisfaisant sera donc vraisemblablement difficile, et à vrai dire il
s'agit peut-être de la plus grande difficulté à laquelle notre anonymiseur fait
face.

\newpage

\section*{Conclusion}
\addcontentsline{toc}{section}{Conclusion}



\newpage
\tableofcontents
\end{document}