\documentclass{article}

\usepackage[utf8]{inputenc}
\usepackage[T1]{fontenc}
\usepackage[francais]{babel}

\usepackage{parskip}

\title{Anonymisation d'emails}
\author{Antoine Lafouasse}

\begin{document}
\maketitle

\section*{Introduction}
\addcontentsline{toc}{section}{Introduction}

L'anonymisation de données est une tâche de pré-traitement, qui consiste à
effacer d'une donnée textuelle toute mention ou élément pouvant servir à
identifier une entité du monde réel ; elle sert de fait à sécuriser une donnée
au sens où il devient plus difficile --- idéalement impossible --- de remonter
aux origines des données.

Elle joue ainsi un rôle essentiel en TAL, dans la mesure où des données
anonymisées peuvent être plus facilement exploitées en tant que corpus sans
lever de problèmes d'éthique : des données brutes, sans anonymisation, posent
le problème de l'utilisation de données potentiellement sensibles et
personnelles de personnes physiques ou morales, sans autorisation de leur part.
L'anonymisation de données en TAL est ainsi portée par une motivation autant
éthique que juridique.

Cette étude se propose de présenter une solution logicielle visant à anonymiser
un corpus écrit, consistant en données issues d'emails.

\section{Présentation des données}

\section{Description de la procédure utilisée}

\section{Analyse des limites et résultats}

\section*{Conclusion}
\addcontentsline{toc}{section}{Conclusion}

\newpage
\tableofcontents
\end{document}